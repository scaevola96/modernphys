\documentclass[a4paper,12pt]{article}
\usepackage{amsmath,amssymb,amsfonts,amsthm}
\usepackage{tikz}
\usepackage [T2A] {fontenc} 
\usepackage[russian]{babel}
\usepackage{cmap} 
\usepackage{ gensymb }
\usepackage[utf8x]{inputenc}
\usepackage[unicode]{hyperref}
\usepackage{ textcomp }
\usepackage{datetime}
\usepackage{physics}
\usepackage{float}
\usepackage{cancel}
\usepackage{mathtools}
\usepackage[margin=0.7in]{geometry}
\usepackage{fancyhdr}
\pagestyle{fancy}

\newcommand{\func}[2]{\left\langle #1, #2\right\rangle}
\newcommand{\Sw}[1]{\mathcal S \left( #1 \right)}
\newcommand{\Ft}[1]{\mathcal F\left[ #1 \right]}
\newcommand{\sign}{\text{sign}}
\newcommand{\schgl}{\mathcal{S'}(\mathbb{R}^l)}
\newcommand{\RR}{\mathbb{R}}
\newcommand{\sch}{\mathcal{S}(\mathbb{R}^m)} 

\newcommand{\paper}[2]{ 
\newpage
\subsection*{#1. #2}
\addcontentsline{toc}{subsection}{#1}
}

\makeatletter

\fancyhead[L]{\footnotesize Основы современной физики}
\fancyhead[RO]{Последняя компиляция: \today \, \currenttime}
\fancyfoot[R]{\thepage}
\fancyfoot[C]{}

\title{Основы современной физики}
\author{ЛФИ им. Ландау}
\date{Весенний семестр 2019 г.}

\renewcommand{\maketitle}{%
	\noindent{\bfseries\scshape\large\@title\ \mdseries\upshape}\par
	\noindent {\large\itshape\@author}
	\vskip 2ex}
\makeatother

\begin{document}

\maketitle

\section{Кристаллические структуры твёрдых тел, трансляционная симметрия кристаллов, решётка Бравэ, элементарная и примитивная ячейки (на примере ГЦК-решётки), базис.}

\subsection{Кристаллическиа структура}
\subsection{Трансляционная симметрия}
\subsection{Решётка и базис}
\subsection{Решётка Бравэ}
\subsection{Элементарная ячейка }
\subsection{Примитивная ячейка}
\section{Рентгеновские и нейтронные методы исследования кристаллических структур, дифракция Вульфа-Брэгга, обратная решётка, зона Бриллюэна.}
\subsection{Кристалл как дифракционная решетка}
\subsection{Условие Брэгга-Вульфа}
\subsection{Дифракция на кристалле}
\subsection{Обратная решётка}
\subsection{Зона Бриллюэна}
\section{Типы связей в кристаллах: кулоновская (атомные кристаллы), ковалентная (обменное взаимодействие), ван-дер-ваальсовская (молекулярные кристаллы), металлическая. Потенциал Леннард-Джонса.}

\subsection{Потенциальная энергия взаимодействия двух атомов}
\subsection{Основные типы связей в кристаллах}
\subsection{Ионная связь}
\subsection{Ковалентная связь}
\subsection{Ван-дер-Ваальсова связь}
\subsection{Металлическая связь}


\section{Дефекты кристаллической решетки. }
\subsection{Дефекты}
\subsection{Типы точечных дефектов}
\subsection{Свойства точечных дефектов}
\subsection{Типы линейных дефектов}
\subsection{Типы плоских дефектов}
\subsection{Колебаняи моноатомной цепочки. понятие о квазиимпульсе. Дискретность квазиимпульса как следствие периодических грничных условий.}
\subsection{Колебания моноатомной цепочки}
\subsection{Понятие о квазиимпульсе}
\subsection{Дискретность квазиимпульса как следствие периодических граничных условий}
\section{Колебание двухатомной цепочки, акустическая и оптическая ветви колебаний}
\subsection{Колебания двухатомной цепочки}
\subsection{Акустическая и оптическая ветви колебаний}
\end{document}